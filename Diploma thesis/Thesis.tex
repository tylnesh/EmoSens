\documentclass[11pt,a4paper]{report}
\usepackage[cp1250]{inputenc}
\usepackage[english]{babel}
\usepackage[IL2]{fontenc}
\usepackage{amsmath}
\usepackage{amsfonts}
\usepackage{amssymb}
\usepackage{makeidx}
\usepackage{graphicx}
\usepackage{lmodern}
\usepackage[left=3.5cm,right=2cm,top=2cm,bottom=2cm]{geometry}
\author{Bc. Michal Koh�tek}
\title{Recognizing the user's emotional state using intelligent control systems}

\begin{document}
\begin{titlepage}
{\centering
{\bfseries\LARGE UNIVERZITA KON�TANT�NA FILOZOFA V NITRE \par}
{\bfseries\LARGE FAKULTA PR�RODN�CH VIED \par}
\vfill
{\bfseries\LARGE RECOGNIZING THE USER'S EMOTIONAL STATE USING INTELLIGENT CONTROL SYSTEMS\par}
\vspace{1cm}
{\bfseries\Large Diploma thesis\par}
\par}
\vfill
\Large{
Study programme:\par
Field of Study:\par
Department:\par
Supervisor: Mgr. Martin Magdin, PhD\par}
\vfill
\textbf{Nitra 2018}\hfill\textbf{Bc. Michal Koh�tek}
\end{titlepage}

\chapter{Introduction}

\begin{flushright}

"Smiles are probably the most underrated \\ facial expressions, 
 much more complicated \\ than most people realize. 
 There are dozens\\ of smiles, each differing in appearance \\
 and in the message expressed."\\
 - Paul Ekman
\end{flushright}

https://www.ted.com/talks/tiffany_watt_smith_the_history_of_human_emotions/footnotes?referrer=playlist-what_are_emotions#t-842147

\paragraph{} Emotions are at the core of the human experience, albeit very hard to define, recognize and name, even in yourself. They are, by definition different from person to person, in diverse cultures and upbringings. Our perception of emotions and their classification has evolved in recent years. Various authors has tried to divide our emotional states into basic categories such as Ekman's Anger, Contempt, Fear, Disgust, Happiness, Sadness and Surprise. However, recent work by psychologists and historians alike show, that a more complex look at emotions might be needed. 

\paragraph{}Emotional recognition changes with time. In 12th century, bards looked at yawning not as a sign of boredom or tiredness, but as a sign of a hidden and deep love. Early Christians recognized an emotion called "accidie", a lethargy and despair brought about by flying demons. Boredom, as such was first really  felt by the Victorians as a response to the new ideas of leisure time and self-improvement. Among the psychologist, there is a standing question whether some cultures feel some hard to define emotions more strongly, because they bothered to name them as separate kinds. For example the Russian "toska", a longing with nothing to long for, as coined by Nabokov. Recent developments of cognitive science tell us, that emotions are not just simple reflexes, but inherently complex and elastic systems of response towards both the biologies that we've inherited and the cultures, that we live in now. They are not just simple chemistry, but a cognitive phenomena, not shaped only by our body functions, but also our thought process, concepts and language. The neuroscientist Lisa Feldman Barett (TODO: citation needed) studies this dynamic relationship between words and emotions. She argues, that when a person learns a new word for an emotion, they also learn to feel and recognize it. There is a historicity to emotions, they have changed in history, often times very dramatically, in response to new cultural expectations, religious beliefs, new ideas about gender, age, ethnicity, economical and political idealogies. There is a push to increase our emotional intelligence. Emotions are so powerful, that in past, they were sometimes thought to be a cause of illness. In 17th century, there was a student attending the Swiss univesity in Basel. He came afflicted with fever, heart palpitations, skin sores on his body and was close to dying. When they sent him back home to die, he started getting better and by the time, he returned to his hometown, he almost entirely recovered. In 1688, Johannes Hofer (citation needed), medical doctor, learnt of this case and many like it and coined the term for a severe homesickness as "nostalgia". Last confirmed death by nostalgia was an American soldier fighting during the First World War in France. In early 20th century, this feeling has morphed more into a longing for lost time, instead of homesickness and downgraded in severity. 

\paragraph{} Nowadays, our culture celebrates happiness, as it is said to make us a better workers, parents and partners. However, in 16th century, this position was filled by sadness, as is evident by self-help books from that period, which tried to encourage sadness in readers by giving them lists of reasons to be disappointedd.

\chapter{Motivation}
\paragraph{}Emotion recognition is an important object of studies in today's psychology, with many potential uses and applications. Correctly assessing and recognizing subject's emotion can lead to better understanding it's motivation and inner working. Data gained through methods described below can be used to assess the effectiveness of marketing, comprehensibility of lectures, usability of user interfaces, impact of therapy, etc. Previous implementations of emotion recognition technology often have had a multitude of disadvantages, which prohibited it's daily and widespread usage.
 
\chapter{Classifications of emotions}
\chapter{Types of emotions}
\chapter{Physiological markers of emotions}
\chapter{Hardware sensors}
\chapter{Software technologies}



\end{document}