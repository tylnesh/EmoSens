\documentclass[11pt,a4paper]{report}
\usepackage[cp1250]{inputenc}
\usepackage[english]{babel}
\usepackage[IL2]{fontenc}
\usepackage{amsmath}
\usepackage{amsfonts}
\usepackage{amssymb}
\usepackage{makeidx}
\usepackage{graphicx}
\usepackage{lmodern}
\usepackage[left=3.5cm,right=2cm,top=2cm,bottom=2cm]{geometry}
\author{Bc. Michal Koh�tek}
\title{Recognizing the user's emotional state using intelligent control systems}

\begin{document}
\begin{titlepage}
{\centering
{\bfseries\LARGE UNIVERZITA KON�TANT�NA FILOZOFA V NITRE \par}
{\bfseries\LARGE FAKULTA PR�RODN�CH VIED \par}
\vfill
{\bfseries\LARGE RECOGNIZING THE USER'S EMOTIONAL STATE USING INTELLIGENT CONTROL SYSTEMS\par}
\vspace{1cm}
{\bfseries\Large Diploma thesis\par}
\par}
\vfill
\Large{
Study programme:\par
Field of Study:\par
Department:\par
Supervisor: Mgr. Martin Magdin, PhD\par}
\vfill
\textbf{Nitra 2018}\hfill\textbf{Bc. Michal Koh�tek}
\end{titlepage}

\chapter{Introduction}
\chapter{Motivation}
\paragraph{}Emotion recognition is an important object of studies in today's psychology, with many potential uses and applications. Correctly assessing and recognizing subject's emotion can lead to better understanding it's motivation and inner working. Data gained through methods described below can be used to assess the effectiveness of marketing, comprehensibility of lectures, usability of user interfaces, impact of therapy, etc. Previous implementations of emotion recognition technology often have had a multitude of disadvantages, which prohibited it's daily and widespread usage.
 
\chapter{Classifications of emotions}
\chapter{Types of emotions}
\chapter{Physiological markers of emotions}
\chapter{Hardware sensors}
\chapter{Software technologies}



\end{document}